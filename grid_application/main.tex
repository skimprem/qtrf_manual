\documentclass[12pt,a4paper,draft,final]{extarticle}
\usepackage[utf8]{inputenc}
\usepackage[russian]{babel}
\usepackage[OT1]{fontenc}
\usepackage{amsmath}
\usepackage{amsfonts}
\usepackage{amssymb}
\usepackage{makeidx}
\usepackage{graphicx}
\usepackage{hyperref}
\usepackage{subcaption}
\usepackage{listings}
\usepackage{xcolor}
\usepackage{verbatim}
\usepackage[left=2cm,right=2cm,top=2cm,bottom=2cm]{geometry}

\sloppy

% Цветовые схемы
\definecolor{codegreen}{rgb}{0,0.6,0}
\definecolor{codeblue}{rgb}{0,0,0.8}
\definecolor{codegray}{rgb}{0.5,0.5,0.5}
\definecolor{backcolor}{rgb}{0.95,0.95,0.92}  % Светлый фон

% Настройки lstlisting
\lstdefinestyle{mystyle}{
    backgroundcolor=\color{backcolor},  % Фон
    commentstyle=\color{codegreen},     % Комментарии
    keywordstyle=\color{codeblue},      % Ключевые слова
    numberstyle=\tiny\color{codegray},  % Стиль номеров строк
    stringstyle=\color{red},            % Стиль строковых значений
    basicstyle=\ttfamily\footnotesize,  % Основной стиль текста
    breakatwhitespace=false,            % Разрыв по пробелам
    breaklines=true,                     % Автоматический перенос строк
    captionpos=b,                        % Подпись снизу (b - bottom)
    keepspaces=true,                     % Сохранение пробелов
    numbers=left,                        % Номера строк слева
    numbersep=5pt,                       % Отступ номеров строк
    showspaces=false,                    % Не показывать пробелы
    showstringspaces=false,              % Не показывать пробелы в строках
    showtabs=false,                       % Не показывать табуляцию
    tabsize=4                             % Размер табуляции
}

% Применяем стиль ко всем листингам
\lstset{style=mystyle}


\title{Руководство по использованию матрицы трансформации для пересчета координат в ГИС-приложениях}
\author{Роман Сермягин}
\begin{document}
\maketitle

\section{Введение}

В данном документе описывается использование матрицы (далее~-- <<сетка>>)
трансформации для пересчета координат в ГИС-приложениях и стандартном
вводе-выводе командной строки.

Сетка трансформации используется для пересчета геодезических координат из одной
системы координат в другую. В данном документе описывается использование сетки
трансформации для пересчета координат из системы координат СК-42 в систему
координат QazTRF-23.

\section{Описание матрицы трансформации}

Сетка трансформации представляет собой двухполосную регулярную сетку в формате
GeoTIFF, в узлах которой записаны поправки, представляющие собой сдвиг
геодезических координат.

Параметры сетки представлены в табл.~\ref{tab:grid_params}.

\begin{table}[ht!]
    \centering
    \caption{Параметры сетки трансформации}
    \label{tab:grid_params}
    \begin{tabular}{|l|l|}
        \hline
        Параметр & Значение \\
        \hline
        Формат файла                        & GeoTIFF \\
        Название файла                      & \verb|QazTRF_2025xxxx_kz.tif| \\
        Размер ячейки                       & 0.05 \textdegree{} (градусы дуги) \\
        Размер сетки                        & 817\texttimes{}299 \\
        Диапазон                            & 46.475\texttimes{}87.325 \textdegree{} в.~д. \\
                                            & 40.525\texttimes{}55.475 \textdegree{} с.~ш. \\
        Единицы измерения ячеек             & $''$ (секунды дуги) \\
        Минимальное/максимальное значение   & 0.109/1.832 $''$ широты \\ 
                                            & -5.091/-1.248 $''$ долготы \\ 
        Исходная система координат          & СК-42 \\
        Целевая система координат           & QazTRF-23 \\
        \hline
    \end{tabular}
\end{table}

На рис.~\ref{fig:grid_shifts} показаны сдвиги геодезических координат между
системами координат СК-42 и QazTRF-23.

\begin{figure}
    \begin{subfigure}{\textwidth}
        \includegraphics[width=\textwidth]{figs/lat}
        \caption{Полоса 1: сдвиг по широте}
    \end{subfigure}
    \hfill
    \begin{subfigure}{\textwidth}
        \includegraphics[width=\textwidth]{figs/lon}
        \caption{Полоса 2: сдвиг по долготе}
    \end{subfigure}
    \caption{Сдвиги геодезических координат}
    \label{fig:grid_shifts}
\end{figure}

\section{Использование сетки трансформации}

Использование сетки трансформации заключается в получение поправок в
географические координаты для любой точки, в пределах которой находится сетка.
Поправки вычисляются методом интерполяции, основанным на значениях узлов сетки.
Как правило, все современные ГИС-приложения, поддерживают такой тип
трансформации.

Ниже представлены инструкции по установке сетки трансформации в систему и ее
использованию в ГИС-приложении QGIS и командной строке.


\subsection{Установка сетки в систему}
\label{ssec:grid_setup}

Одним из наиболее удобных способов применения сетки трансформации является
использование библиотеки PROJ~\cite{evendenPROJ2024}, на основе которой работают
многие ГИС-приложения, включая QGIS~\cite{QGIS_software}.

\subsubsection{Установка сетки в систему Windows}

Для установки сетки трансформации в систему Windows выполните следующие шаги:
\begin{enumerate}

    \item Убедитесь, что ГИС-приложение QGIS установлено на вашем компьютере и
    корректно работает. 
    
    \item Скопируйте файл сетки \lstinline|qazgrid.tif| трансформации в каталог
    модуля \verb|proj|. В зависимости от версии QGIS и способа установки, путь к
    каталогу \verb|proj| может отличаться, например:
    
    \lstinline|C:\Program Files\QGIS 3.xx.x\share\proj\|

    или

    \lstinline|C:\OSGeo4W\share\proj\|

\end{enumerate}

\subsubsection{Установка сетки в систему GNU/Linux}

Для установки сетки трансформации в систему GNU/Linux выполните следующие шаги:
\begin{enumerate}

    \item Убедитесь, что ГИС-приложение QGIS установлено на вашем компьютере и
    корректно работает. 
    
    \item Скопируйте файл сетки \lstinline|qazgrid.tif| трансформации в каталог
    модуля \verb|proj|. В зависимости от версии QGIS и способа установки, путь к
    каталогу \verb|proj| может отличаться, например:
    
    \lstinline|/usr/share/proj/|

\end{enumerate}

\subsection{Использование сетки в ГИС-приложении QGIS}

Применение параметров трансформации в ГИС-приложении QGIS выполняется <<на
лету>>. Для этого для каждого слоя должны быть заданы параметры проекции,
соответствующие координатам его геометрии. Координатной системой назначения
является текущая система координат, выбранная для проекта.

Ниже представлены инструкции по созданию пользовательских проекций на основе
сетки  трансформации и их использованию в ГИС-приложении QGIS.

\subsubsection{Настройка пользовательской системы координат}
\label{sssec:custom_crs}

Для использования сетки трансформации в ГИС-приложении QGIS необходимо создать 
пользовательскую систему координат. В зависимости от того, в какой системе
представлены исходные координаты, настройки будут отличаться.

\paragraph{Географические координаты:}

для настройки пользовательской системы, применяемой для исходных географических
координат (широта, долгота) необходимо выполнить следующие шаги:
\begin{enumerate}

    \item Запустите ГИС-приложение QGIS.

    \item Откройте меню <<Настройки>> и выберите один из пунктов:
    <<Пользовательские проекции>> или <<Параметры>>
    \ref{fig:main_parameters_select}.

    \begin{figure}[ht!]
        \centering
        \includegraphics[width=0.7\textwidth]{figs/main_parameters_select}
        \caption{Выбор меню <<Пользовательские настройки>>}
        \label{fig:main_parameters_select}
    \end{figure}

    \item В открывшемся окне выберите пункт <<Заданные пользователем СК>> во
    вкладке <<Системы координат и преобразования>>.
    
    \item Нажмите кнопку <<Добавить>>.

    \item Введите название системы координат, например, <<QazTRF-23>>.

    \item В поле <<Параметры>> ведите параметры проекции в формате
    PROJ.4. или WKT (см. рис.~\ref{fig:parameters_proj_code}):
    \begin{enumerate}
        
        \item В формате PROJ.4:

\begin{lstlisting}
+proj=longlat +ellps=krass +nadgrids=qazgrid.tif +no_defs +type=crs
\end{lstlisting}
    
        \item В формате WKT:

\begin{lstlisting}
BOUNDCRS[
    SOURCECRS[
        GEOGCRS["unknown",
            DATUM["Unknown based on Krassovsky, 1942 ellipsoid using nadgrids=qazgrid.tif",
                ELLIPSOID["Krassovsky, 1942",6378245,298.3,
                    LENGTHUNIT["metre",1,
                        ID["EPSG",9001]]]],
            PRIMEM["Greenwich",0,
                ANGLEUNIT["degree",0.0174532925199433],
                ID["EPSG",8901]],
            CS[ellipsoidal,2],
                AXIS["longitude",east,
                    ORDER[1],
                    ANGLEUNIT["degree",0.0174532925199433,
                        ID["EPSG",9122]]],
                AXIS["latitude",north,
                    ORDER[2],
                    ANGLEUNIT["degree",0.0174532925199433,
                        ID["EPSG",9122]]]]],
    TARGETCRS[
        GEOGCRS["WGS 84",
            DATUM["World Geodetic System 1984",
                ELLIPSOID["WGS 84",6378137,298.257223563,
                    LENGTHUNIT["metre",1]]],
            PRIMEM["Greenwich",0,
                ANGLEUNIT["degree",0.0174532925199433]],
            CS[ellipsoidal,2],
                AXIS["latitude",north,
                    ORDER[1],
                    ANGLEUNIT["degree",0.0174532925199433]],
                AXIS["longitude",east,
                    ORDER[2],
                    ANGLEUNIT["degree",0.0174532925199433]],
            ID["EPSG",4326]]],
    ABRIDGEDTRANSFORMATION["unknown to WGS84",
        METHOD["NTv2",
            ID["EPSG",9615]],
        PARAMETERFILE["Latitude and longitude difference file","qazgrid.tif",
            ID["EPSG",8656]]]]
\end{lstlisting}

    \end{enumerate}

    \begin{figure}[ht!]
        \centering
        \includegraphics[width=0.7\textwidth]{figs/parameters_proj_code}
        \caption{Задание пользовательской системы координат}
        \label{fig:parameters_proj_code}
    \end{figure}

    \item Для проверки корректности введенного кода нажмите <<Проверить>>. При
    отсутствии ошибок выйдет сообщение <<Описание проекции в формате Proj
    корректно.>> (см. рис.~\ref{fig:correct}).
    
    \begin{figure}[ht!]
        \centering
        \includegraphics[width=0.4\textwidth]{figs/correct}
        \caption{Проверка кода}
        \label{fig:correct}
    \end{figure}
    
    \item Для проверки корректности работы параметров проекции, раскройте
    вкладку <<Проверить>> и введите координаты в соответствующие поля <<широта>>
    и <<долгота>> (см. рис.~\ref{fig:check_params}).

    \begin{figure}[ht!]
        \centering
        \includegraphics[width=0.7\textwidth]{figs/check_params}
        \caption{Проверка параметров проекции}
        \label{fig:check_params}
    \end{figure}

\end{enumerate}

\paragraph{Координаты в проекции Гаусса-Крюгера:}

для настройки пользовательской системы, применяемой для исходных координат в
проекции Гаусса-Крюгера необходимо выполнить следующие шаги:

\begin{enumerate}

    \item Запустите ГИС-приложение QGIS.

    \item Откройте меню <<Настройки>> и выберите один из пунктов:
    <<Пользовательские проекции>> или <<Параметры>>.

    \item В открывшемся окне во вкладке <<Системы координат и преобразования>>
    выберите пункт <<Заданные пользователем СК>>.

    \item Нажмите кнопку <<Добавить>>.

    \item Введите название системы координат, например, <<QazTRF-23 Gauss-Kruger
    Zone $ZZ$>>, где $ZZ$~-- номер зоны, который определяется по первым цифрам
    горизонтальных координат точки.

    \item В поле <<Параметры>> ведите параметры проекции в формате
    PROJ.4. или WKT, например:
    \begin{enumerate}
        \item В формате PROJ.4:

\begin{lstlisting}
+proj=tmerc +lat_0=0 +lon_0=XX +k=1 +x_0=ZZ500000 +y_0=0 +ellps=krass
+nadgrids=qazgrid.tif +units=m +no_defs +type=crs
\end{lstlisting}

где \lstinline{XX}~-- осевой меридиан, вычисляемый по формуле
$XX = ZZ \cdot 6 - 3$, а \lstinline{+x_0=ZZ500000}~-- сдвиг 500
км на запад от осевого меридиана для данной зоны. Например, для зоны $ZZ = 10$
осевой меридиан \lstinline|+lon_0=57|, а сдвиг \lstinline|+x_0=10500000|.

        \item В формате WKT:

\begin{lstlisting}
BOUNDCRS[
    SOURCECRS[
        PROJCRS["unknown",
            BASEGEOGCRS["unknown",
                DATUM["Unknown based on Krassovsky, 1942 ellipsoid using nadgrids=qazgrid.tif",
                    ELLIPSOID["Krassovsky, 1942",6378245,298.3,
                        LENGTHUNIT["metre",1,
                            ID["EPSG",9001]]]],
                PRIMEM["Greenwich",0,
                    ANGLEUNIT["degree",0.0174532925199433],
                    ID["EPSG",8901]]],
            CONVERSION["unknown",
                METHOD["Transverse Mercator",
                    ID["EPSG",9807]],
                PARAMETER["Latitude of natural origin",0,
                    ANGLEUNIT["degree",0.0174532925199433],
                    ID["EPSG",8801]],
                PARAMETER["Longitude of natural origin",57,
                    ANGLEUNIT["degree",0.0174532925199433],
                    ID["EPSG",8802]],
                PARAMETER["Scale factor at natural origin",1,
                    SCALEUNIT["unity",1],
                    ID["EPSG",8805]],
                PARAMETER["False easting",10500000,
                    LENGTHUNIT["metre",1],
                    ID["EPSG",8806]],
                PARAMETER["False northing",0,
                    LENGTHUNIT["metre",1],
                    ID["EPSG",8807]]],
            CS[Cartesian,2],
                AXIS["(E)",east,
                    ORDER[1],
                    LENGTHUNIT["metre",1,
                        ID["EPSG",9001]]],
                AXIS["(N)",north,
                    ORDER[2],
                    LENGTHUNIT["metre",1,
                        ID["EPSG",9001]]]]],
    TARGETCRS[
        GEOGCRS["WGS 84",
            DATUM["World Geodetic System 1984",
                ELLIPSOID["WGS 84",6378137,298.257223563,
                    LENGTHUNIT["metre",1]]],
            PRIMEM["Greenwich",0,
                ANGLEUNIT["degree",0.0174532925199433]],
            CS[ellipsoidal,2],
                AXIS["latitude",north,
                    ORDER[1],
                    ANGLEUNIT["degree",0.0174532925199433]],
                AXIS["longitude",east,
                    ORDER[2],
                    ANGLEUNIT["degree",0.0174532925199433]],
            ID["EPSG",4326]]],
    ABRIDGEDTRANSFORMATION["unknown to WGS84",
        METHOD["NTv2",
            ID["EPSG",9615]],
        PARAMETERFILE["Latitude and longitude difference file","qazgrid.tif",
            ID["EPSG",8656]]]]
\end{lstlisting}

    \end{enumerate}

\begin{figure}[ht!]
    \centering
    \includegraphics[width=0.7\textwidth]{figs/crs_tmerc}
    \caption{Задание пользовательской системы координат в проекции Гаусса-Крюгера}
    \label{fig:crs_tmerk}
\end{figure}

    \item Для проверки корректности введенного кода нажмите <<Проверить>>. При
    отсутствии ошибок выйдет сообщение <<Описание проекции в формате Proj
    корректно>>.

    \item Для проверки корректности работы параметров проекции, раскройте
    вкладку <<Проверить>> и введите координаты в соответствующие поля <<Широта>> и
    <<Долгота>>. В полях <<Целевая СК>> должны появиться координаты в проекции
    Гаусса-Крюгера.



\end{enumerate}

\subsubsection{Использование пользовательской системы координат}

\begin{figure}[ht!]
    \centering
    \includegraphics[width=0.7\textwidth]{figs/select_crs}
    \caption{Выбор пользовательской системы координат}
    \label{fig:select_crs}
\end{figure}

При загрузке данных из меню <<Добавить слой>>, исходная система координат либо
определяется самостоятельно (например, при загрузке из файла формата GeoJSON или
Shapefile), либо задается вручную (например, при загрузке данных из файла
формата CSV).

Для применения пользовательских параметров проекции с использованием сетки
(см. разд.~\ref{sssec:custom_crs}) необходимо выбрать при загрузке созданную систему
координат, коответствующую координатам загружаемых данных.

Для этого в меню <<Выбор системы координат>> должен быть активен пункт
<<Предустановленные СК>>, а в разделе <<Предопределенные системы координат>> в
списке <<Пользовательская>> должна быть выбрана созданная система координат (см.
рис.~\ref{fig:select_crs}).

Например, при загрузке данных из CSV-файла с координатами в проекции
Гаусса-Крюгера попадающими в зону 10, необходимо выбрать созданную систему
координат QazTRF-23 Gauss-Kruger Zone 10. 

При необходимости пересчета координат в другую систему уже загруженного слоя,
необходимо изменить систему координат слоя. Для этого из контекстного меню слоя
необходимо выбрать пункт <<Система координат слоя>>, далее <<Изменить систему
координат слоя>> (см. рис.~\ref{fig:change_crs}). В открывшемся окне выбрать
созданную систему координат (см.  рис.~\ref{fig:select_crs}).

\begin{figure}[ht!]
    \centering
    \includegraphics[width=0.5\textwidth]{figs/change_crs}
    \caption{Изменение системы координат слоя}
    \label{fig:change_crs}
\end{figure}

Например, если изначально слой точек в проекции Гаусса-Крюгера и зоной
10 был загружен со стандартными параметрами \lstinline|EPSG:28410 -- Pulkovo 1942 / Gauss-Kruger zone 10|, соответствующими
ГОСТ~32453-2017~\cite{GOST324532017Globalnaya2017}, то их разность с
фактическими координатами, измеренными в QazTRF-23, будет составлять несколько
метров (см. рис.~\ref{fig:before}). После применения пользовательской проекции
QazTRF-23 Gauss-Kruger Zone 10 расхождение между точками будет в пределах 10 см
(см. рис.~\ref{fig:after}).

\begin{figure}[ht!]
    \centering
    \begin{subfigure}{0.49\textwidth}
        \includegraphics[width=\textwidth]{figs/compare_befor}
        \caption{Стандартная проекция: разность между точками $\approx$~7.2~м}
        \label{fig:before}
    \end{subfigure}
    \hfill
    \begin{subfigure}{0.49\textwidth}
        \includegraphics[width=\textwidth]{figs/compare_after}
        \caption{Пользовательская проекция: разность между точками $\approx$~0.1~м}
        \label{fig:after}
    \end{subfigure}
    \caption{Сравнение координат точек в разных системах координат}
    \label{fig:compare}
\end{figure}

\subsection{Использование сетки в командной строке}

Для использования сетки трансформации в командной строке необходимо установить
пакеты PROJ. При установке QGIS этот пакет будет установлен автоматически.

\subsubsection{Пример использования сетки в командной строке Windows}

Для примера рассмотрим пересчет координат точки из СК-42 в QazTRF-23 в системе
Windows. Например, исходные координаты точек записаны в текстовые файлы
\lstinline|input_lonlat.txt| и \lstinline|input_tmerc.txt| в форматах <<Долгота
Широта Название>> и <<Восток Север Название>>, соответственно:
\lstinputlisting[caption={Содержание файла
\lstinline|input_lonlat.txt| в географических координатах}]{test_data/input_lonlat.txt}
\lstinputlisting[caption={Содержание файла
\lstinline|input_tmerc.txt| в координатах проекции Гаусса-Крюгера}]{test_data/input_tmerc.txt}

Для этого выполните следующие шаги:
\begin{enumerate}

    \item Убедитесь, что сетка трансформации \lstinline|qazgrid.tif| установлена
    в систему (см. разд.~\ref{ssec:grid_setup}).

    \item Откройте командную строку OSGeo4W. Для этого нажмите <<Пуск>>, далее в
    меню <<Приложения>> в папке <<OSGeo4W>> выберите <<OSGeo4W Shell>>.
    Откроектся терминал <<OSGeo4W>>.

    \item 

    \item 
\end{enumerate}

% \bibliographystyle{ugost2008}
\bibliographystyle{plain}
\bibliography{grid_application}

\end{document}