\documentclass[12pt,a4paper,draft,final]{extarticle}
\usepackage[utf8]{inputenc}
\usepackage[russian]{babel}
\usepackage[OT1]{fontenc}
\usepackage{amsmath}
\usepackage{amsfonts}
\usepackage{amssymb}
\usepackage{makeidx}
\usepackage{graphicx}
\usepackage{hyperref}
\usepackage{subcaption}
\usepackage{listings}
\usepackage{xcolor}
\usepackage{verbatim}
\usepackage[left=2cm,right=2cm,top=2cm,bottom=2cm]{geometry}

% Цветовые схемы
\definecolor{codegreen}{rgb}{0,0.6,0}
\definecolor{codeblue}{rgb}{0,0,0.8}
\definecolor{codegray}{rgb}{0.5,0.5,0.5}
\definecolor{backcolor}{rgb}{0.95,0.95,0.92}  % Светлый фон

% Настройки lstlisting
\lstdefinestyle{mystyle}{
    backgroundcolor=\color{backcolor},  % Фон
    commentstyle=\color{codegreen},     % Комментарии
    keywordstyle=\color{codeblue},      % Ключевые слова
    numberstyle=\tiny\color{codegray},  % Стиль номеров строк
    stringstyle=\color{red},            % Стиль строковых значений
    basicstyle=\ttfamily\footnotesize,  % Основной стиль текста
    breakatwhitespace=false,            % Разрыв по пробелам
    breaklines=true,                     % Автоматический перенос строк
    captionpos=b,                        % Подпись снизу (b - bottom)
    keepspaces=true,                     % Сохранение пробелов
    numbers=left,                        % Номера строк слева
    numbersep=5pt,                       % Отступ номеров строк
    showspaces=false,                    % Не показывать пробелы
    showstringspaces=false,              % Не показывать пробелы в строках
    showtabs=false,                       % Не показывать табуляцию
    tabsize=4                             % Размер табуляции
}

% Применяем стиль ко всем листингам
\lstset{style=mystyle}


\title{Руководство по использованию матрицы трансформации для пересчета координат в ГИС-приложениях}
\author{Роман Сермягин}
\begin{document}
\maketitle

\section{Введение}

В данном документе описывается использование матрицы (далее~-- <<сетка>>)
трансформации для пересчета координат в ГИС-приложениях и стандартном
вводе-выводе командной строки.

Сетка трансформации используется для пересчета геодезических координат из одной
системы координат в другую. В данном документе описывается использование сетки
трансформации для пересчета координат из системы координат СК-42 в систему
координат QazTRF-23.

\section{Описание матрицы трансформации}

Сетка трансформации представляет собой двухполосную регулярную сетку в формате
GeoTIFF, в узлах которой записаны поправки, представляющие собой сдвиг
геодезических координат.

Параметры сетки представлены в табл.~\ref{tab:grid_params}.

\begin{table}[ht!]
    \centering
    \caption{Параметры сетки трансформации}
    \label{tab:grid_params}
    \begin{tabular}{|l|l|}
        \hline
        Параметр & Значение \\
        \hline
        Формат файла                        & GeoTIFF \\
        Название файла                      & \verb|QazTRF_2025xxxx_kz.tif| \\
        Размер ячейки                       & 0.05 \textdegree{} (градусы дуги) \\
        Размер сетки                        & 817\texttimes{}299 \\
        Диапазон                            & 46.475\texttimes{}87.325 \textdegree{} в.~д. \\
                                            & 40.525\texttimes{}55.475 \textdegree{} с.~ш. \\
        Единицы измерения ячеек             & $''$ (секунды дуги) \\
        Минимальное/максимальное значение   & 0.109/1.832 $''$ широты \\ 
                                            & -5.091/-1.248 $''$ долготы \\ 
        Исходная система координат          & СК-42 \\
        Целевая система координат           & QazTRF-23 \\
        \hline
    \end{tabular}
\end{table}

На рис.~\ref{fig:grid_shifts} показаны сдвиги геодезических координат между
системами координат СК-42 и QazTRF-23.

\begin{figure}
    \begin{subfigure}{\textwidth}
        \includegraphics[width=\textwidth]{lat}
        \caption{Полоса 1: сдвиг по широте}
    \end{subfigure}
    \hfill
    \begin{subfigure}{\textwidth}
        \includegraphics[width=\textwidth]{lon}
        \caption{Полоса 2: сдвиг по долготе}
    \end{subfigure}
    \caption{Сдвиги геодезических координат}
    \label{fig:grid_shifts}
\end{figure}

\section{Использование сетки трансформации в QGIS}

\subsection{Установка сетки в систему Windows}

Для установки сетки трансформации в систему Windows выполните следующие шаги:
\begin{enumerate}

    \item Убедитесь, что ГИС-приложение QGIS установлено на вашем компьютере и
    корректно работает. 
    
    \item Скопируйте файл сетки \lstinline|qazgrid.tif| трансформации в каталог
    модуля \verb|proj|. В зависимости от версии QGIS и способа установки, путь к
    каталогу \verb|proj| может отличаться, например:
    
    \lstinline|C:\Program Files\QGIS 3.xx.x\share\proj\|

    или

    \lstinline|C:\OSGeo4W\share\proj\|

\end{enumerate}

\subsection{Настройка пользовательской системы координат}

Для использования сетки трансформации в ГИС-приложении QGIS необходимо создать 
пользовательскую систему координат. В зависимости от того, в какой системе
представлены исходные координаты, настройки будут отличаться.

\subsubsection{Географические координаты}

Для настройки пользовательской системы, применяемой для исходных географических
координат (широта, долгота) необходимо выполнить следующие шаги:
\begin{enumerate}

    \item Запустите ГИС-приложение QGIS.

    \item Откройте меню <<Настройки>> и выберите пункт <<Пользовательские
    проекции>> или <<Параметры>>.

    \item В открывшемся окне выберите пункт <<Заданные пользователем СК>> во
    вкладке <<Системы координат и преобразования>>.
    
    \item Нажмите кнопку <<Добавить>>.

    \item Введите название системы координат, например, <<QazTRF-23>>.

    \item В поле <<Параметры>> ведите параметры сетки трансформации в формате
    PROJ.4. или WKT:
    \begin{enumerate}
        
        \item В формате PROJ.4:
\begin{lstlisting}
+proj=longlat +ellps=krass +nadgrids=qazgrid.tif +no_defs +type=crs
\end{lstlisting}
    
        \item В формате WKT:

\begin{lstlisting}
BOUNDCRS[
    SOURCECRS[
        GEOGCRS["unknown",
            DATUM["Unknown based on Krassovsky, 1942 ellipsoid using nadgrids=qazgrid.tif",
                ELLIPSOID["Krassovsky, 1942",6378245,298.3,
                    LENGTHUNIT["metre",1,
                        ID["EPSG",9001]]]],
            PRIMEM["Greenwich",0,
                ANGLEUNIT["degree",0.0174532925199433],
                ID["EPSG",8901]],
            CS[ellipsoidal,2],
                AXIS["longitude",east,
                    ORDER[1],
                    ANGLEUNIT["degree",0.0174532925199433,
                        ID["EPSG",9122]]],
                AXIS["latitude",north,
                    ORDER[2],
                    ANGLEUNIT["degree",0.0174532925199433,
                        ID["EPSG",9122]]]]],
    TARGETCRS[
        GEOGCRS["WGS 84",
            DATUM["World Geodetic System 1984",
                ELLIPSOID["WGS 84",6378137,298.257223563,
                    LENGTHUNIT["metre",1]]],
            PRIMEM["Greenwich",0,
                ANGLEUNIT["degree",0.0174532925199433]],
            CS[ellipsoidal,2],
                AXIS["latitude",north,
                    ORDER[1],
                    ANGLEUNIT["degree",0.0174532925199433]],
                AXIS["longitude",east,
                    ORDER[2],
                    ANGLEUNIT["degree",0.0174532925199433]],
            ID["EPSG",4326]]],
    ABRIDGEDTRANSFORMATION["unknown to WGS84",
        METHOD["NTv2",
            ID["EPSG",9615]],
        PARAMETERFILE["Latitude and longitude difference file","qazgrid.tif",
            ID["EPSG",8656]]]]
\end{lstlisting}
    
    \end{enumerate}

\end{enumerate}

\end{document}